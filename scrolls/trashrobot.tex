
Trash robot is rainbow and googley eyes.   Trash robot is geometric constructions from cardboard with rainbow duct tape.   Trash robot is felt cutouts on black cotton flannel or sweat pants material.  Trash robot is post apocalyptic trash goblin energy.   Trash robot is chaos.  Trash robot is anarchist science. It's dirty kids in a school bus camper. It's a party by a dumpster.  It's running around in some steam tunnels.  It's weird art dropped in unexpected places. And of course it's robots built from trash.  When in doubt, add more rainbows, more googly eyes, more geometric constructions, and more trash and sticks and rocks.

Trash robot is an open brand.  All these aesthetic properties combined together can be very easy to recognize and replicate, but at the same time they are very clearly outside of any existing copyrights or trademarks.  By producing a large number of derivative works all in the public domain, we force the whole aesthetic to stay permanently in the public domain. If we build this into a powerful and valuable brand, that injects value into the public domain, which can be used to transmit all the rest of our technology also in the public domain. 

This whole structure is in analogy to how brands work in our existing economic system.  We aim to create a brand which stimulates free replication of free things in the same way a company uses their brand to sell their stuff.  This is not like the logos for an open source software project, which are generally property of a non-profit corporation.  Trash robot is not a logo. It is a general aesthetic, like vaporwave or goblincore, which are simply not owned by anyone.  If a big company wants to coopt it, let them.   As long as it stimulates replication of our system, we gain from it.  Attention will stimulate replication, so that's what we want.  