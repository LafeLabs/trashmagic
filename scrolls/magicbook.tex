The magic book is a book which you co-create with other readers and writers.  It can replicate freely along the Geometron network, as described in the Book of Geometron.  This is not a specific book so much as a collection of methods of creating self-replicating books.  

The formats of the books are hand drawn and hand bound zines, bound 6x9 hard copy printed by print on demand Lulu press, printed letter size paper in three ring binders, and a web-based format which is how text is edited.  The web books have code in them which allows them to replicate freely from one Geometron server to another.  

As a keeper of any given version you will have received it from someone who can show you how to pass it along.  You can just pass along a book from one person to another and track it in the Street Network chapter. 

The electronic magic book can be part of an event, including a virtual one, where people message additions to a document to a human operator who edits the document in real time and shares it over a web server with everyone, who can then replicate it again, edit it again and so on.  This can be done on a Raspberry Pi, the cheap open source computer platform used for everything in the Geometron network. 

The Magic book can be used for many things. It can be used to co-create a story.  It can be used to create a shared book documenting a local community. It can be fiction or non-fiction.  It can be many hundreds or even thousands of pages, or just a single page with a short list of links.  It can be entirely images or just text.  What makes it magic is that it can always be replicated and edited by each new person we share it with.  As with trash magic, this fits our working definition of magic: the book replicates the desire to replicate the book.  

The Magic Book can also have oral formats, either from memory in person, over live stream, on a video sharing platform, or on an audio podcast.  Or it can be entirely performance art as we carry out the actions which are encoded in the book, just building and sharing geometry. 
