

We will make everything from trash, without exception. We will abolish
all global supply chains and mining and make things using only the
materials in our immediate environment. Our media will be outside the
property system, merely a shared resource which we use to replicate all
the things in our replication-based economy of trash and sun. However,
all this will take time. Decades. We have to start somewhere.

We must start with the minimum required to have media which is not
property and which can carry the knowledge needed to replicate itself.
That is the Raspberry Pi and the associated systems described in this
work. However, in order for this to get where we need to go, it has to
have a \emph{path} forward to full Trash Magic. We do not have the
resources to do the whole thing at once, nor would it be desirable to do
so if we did, since large intense efforts tend to create systems which
continue to require that form of effort.

The first phase of the network described here is enough to start, to get
natural replication without any external input from grants or startup
capital. The Raspberry Pi can be used to make social media which
generates sufficient value both inside the marketplace of money and
property and outside of it that it should sustain growth. When this
growth starts to ramp up, however, it will pull more and more materials
and energy into the network naturally. The natural next step after the
Raspberry Pi is to start pulling in more mainstream hardware platforms.
This system only needs a web server and the language PHP to work and it
can be replicated. This can be done on Windows, MacOS, iOS, Android, and
any type of Linux. So ultimately the number of servers can get up into
the billions once these start getting switched from private to public
use.

Any given server added to our system of common media adds a certain
amount of value to a local community. When that amount of value is
perceived as being greater than the price of a thing or the value of the
thing to its owner, people will naturally start moving hardware from the
property system to the Geometron network. As with all network growth
dynamics, the value will continue to increase exponentially with the
number of elements on the network. As we cross more and more thresholds
of value, more and more hardware will transfer over.

Of course, the Raspbery Pi and all these commercial off the shelf
machines all still come from a mine. They all still have a finite life
time and are deliberately designed to be unusable after that lifetime.
In some sense using them doesn't solve any real problem of Trash Magic
as they still are all bought for money from an unsustainable system. The
next step in our path of conversion is to start using a wider range of
hardware so that more and more waste can be used in our system which was
on its way to the landfill. To do this, we must turn to the ``Internet
of Things''. This jargon term is used to denote putting computers in
poorly designed products which should not have computers in them.
Generally they solve no real problem and are designed to break as fast
as possible with no possibility to repair. However they are a fantastic
resource for us, since they all have to have basic Internet capability
by definition.

Our task with these devices is to develop processes which are easy to
replicate which remove everything that is not needed from these
machines, remove all proprietary software and all the hardware other
than the basic Internet connectivity parts, and put a stripped down
operating system the sole purpose of which is to host Geometron
documents. As with all tasks like this, the way to do this is with a
fork of Linux based on other forks designed for this kind of task. A
perfect example of this already widely in use is OpenWrt, which is
widely used for making routers useful for all sorts of things.

The path above is sufficient to build our global media network outside
the system of property, which can be used to replicate things made from
trash, while still feeding off the existing extraction based system. We
now turn our attention to the ultimate goal of freedom from that system.

All of our modern electronics comes from a vast and powerful web of
supply which is very centralized, very brittle, and very unsustainable.
Huge quantities of sand with special properties are extracted from the
few places in the world where they can be found and transported with oil
thousands of miles to the few places in the world where microfabrication
happens. This sand is melted in specialized furnaces into giant crystals
of insanely pure silicon, which is sliced into wafers. The wafers are
then put through a mass production process where each wafer has
thousands of chips and each chip has billions of transistors and other
components. These combine to make giant arithmetic engines which have
clocks pulsing as fast as possible, generally billions of times a
second. All the functions of our ``computer systems'' are based on this
very fast arithmetic engine. This all looks a lot like replication, with
many dice being stamped out which are identical to one another, but it's
not quite the same. They are replicated, but the whole system is not
freely replicated. Intellectual property, control of supply chains, and
access to the vast amount of capital required to build these multi
billion dollar facilities all make it so that the \emph{system} is
designed not to replicate.

We must build a new fabrication system from the ground up if we are to
free ourselves from the mine and oil system. To do this, we rethink the
purpose of the machines we are building. We do not want to do
arithmetic. We want to do geometry. We want our machines to do the
absolute minimum work required to make the things we want to make and to
display information to humans. Furthermore, everything we are building
is designed to be controlled directly by humans. When a device is not
being used and not carrying out a fabrication task it should do
\emph{nothing}, take no energy at all. When a media machine is
displaying a static document, the pixels should be energized, and
nothing else. All action is initiated by a human, and leads to response
not based on clocks but on direct sequences of actions which take as
long as they take.

We do not realize how large the inefficiencies are in our current
systems because they have seemingly blown up overnight and we have
nothing to compare them to. Also, they grew up in parallel with advances
in the fundamental science, so we have no perspective on what is
possible with today's science given advances in understanding of things
like organic semiconductors or various exotic materials which did not
exist in the mid 20th century as the current system was evolving. Also,
the growth of the current system involved vast amounts of extraction to
get the various special atoms needed to create novel electronic devices.

But our whole situation is vastly different now than in the 20th
century! Because of the trash feed, we now have every exotic atom
available locally to every single person on Earth. Wherever any of us
are right now at this moment, there is a pile of electronic junk which
is all identical in its atomic composition and which has every exotic
element from antimony to zirconium not only available but in a well
understood and perfectly repeated format. So while it might be
\emph{difficult} to figure out how to use some exotic atom found in a
trashed television, in a replication based economy we only need to do it
once and then push that media out to the global feed and the whole world
gets it for free with no supply chain at all. This is a situation
totally unlike any that has ever existed in human history! We cannot
possibly overstate the power of this situation. Even if our civilization
totally collapses, any future civilization which evolves will always
take this as a starting point, will not have to mine to create whatever
they create. The product of a globalized consumer society cannot be
undone: the redistribution of atomic wealth is permanent.

Also, our economic constraints are totally different than those under
which the technology we wish to replace were constructed. The economic
forces which created our existing microelectronic fabrication systems
always favor size, power, and speed, at the expense of the ability to
repair anything. The faster the chips produced go to the landfill, the
more money the factories make. The more money they make, the bigger they
can get, which lowers price, which makes them go faster and so on. We
now aim to break that cycle and build much simpler and slower things
which we can repair indefinitely. If an artifact is intended to
effectively last forever, being repaired and repurposed and reused
indefinitely, with all the atoms staying in a physical locality
permanently, we want to shift from mass production to craft production.
We no longer need chips to come out of a factory by the billions or to
all be the same and have no defects. If a single circuit takes months to
build and requires a skilled craftsperson, that is acceptable to us as
long as that circuit can be kept in the community for many decades or
centuries. This represents a totally different culture of creation, in
which circuits are made directly by communities for their own benefit,
but based on knowledge which replicates freely across humanity on the
Geometron network.

So what is all this for and how will we build it? We primarily want to
do two things: control the machines that are used for physical
fabrication(printers, 3d printers, laser cutters, milling machines,
lathes, etc.) and control the screens which make up the media. We do all
of this with Geometron. In Geometron, machines manipulate units of
geometric action, and then use those units to do physical things.

We first look at what this means for fabrication machines. In the
current system, all fabrication machines are based on Arduino. The
Arduino is a simple open source hardware system. It is easy to buy,
cheap, and easy to program. To program the Arduino with Geometron we
simply create functions which do geometric actions, and build a
geometric instruction set which controls those. So for example with the
robots built out of broken DVD drives we use for clay fabrication, the
function has actions for moving left, right, up, down, forward, and back
by one unit, actions to double unit and actions to halve unit, and
nothing else. Programs are simply sequences of these actions. Since
these are just geometry, they can also be translated to meaning in a
canvas element of a web browser, and connected with symbols which are
both displayed in the browser and painted on keys to be programmed with
a physical keyboard. This is how we are able to program robots in a web
browser with no arithmetic. The universal nature of geometry allows
geometric programming from a browser to turn into a sequence of actions
on a robot which then turns into physical things, forming a replication
technology of those things.

To make complex things like printed characters in a human language like
English, we also create a Geometron Hypercube which allows some actions
to consist of sequences of actions. So for instance ``draw the letter
A'' will translate to a sequence of ``draw a pixel'' actions, and each
of \emph{those} actions is itself a sequence of actions to move a tool
in whatever way is needed to draw a pixel, like poking a nail into clay
or lowering a drill press to drill a hole. In the Arduino, this
Hypercube is expressed using strings made up of letters. The Arduino,
like all Geometron Virtual Machines, takes a tape of geometric actions
in this case represented by letters and does something, in some cases
another sequence of letters which can in turn call more sequences of
letters, and all of these sequences are just strings. Programs created
in the browser will generally have a text area on the screen which
displays the Arduino code which has the correct strings for the program.
This is then copied into the Arduino, uploaded, and we complete the
connection from browser to machine.

This system can be used to control \emph{any} fabrication tool, and part
of creating our own new way of making things is to close that loop in
all cases. We need to be using pure geometric programming in a web
browser to control the tools which fabricate circuits, metal machines,
metal molds for plastic parts, cut wood parts, moving biological samples
for synthetic biology and indeed \emph{every} fabrication task. All
fabrication is geometry. All geometry can be expressed in sequences of
actions defined using only geometry and denoted to humans using only
symbols also created geometrically in a web browser. Therefore it is
possible to use the Arduino and Raspberry Pi to build up a new type of
web based self-replicating technology.

However, as with the Geometron server, we note that the system based on
Arduino is still reliant on the mine and oil cycle we are trying to
escape. To escape this, we first need the Arduino based fabrication
machines to be making microelectronic circuits. This can be slow. It can
be crude, with large devices initially. But ultimately it is the start
of the process to replace the whole system with full Trash Magic.

What we want from the hardware which replaces the Arduino is just
movements of motors with timers, and a hardware implementation of the
Geometron Hypercube and Geometron Virtual Machine. So we want a physical
medium of some kind to have information encoded in it in a way which
triggers a set of switches which choose what thing happens. This can be
purely mechanical, or combine electrical and mechanical, and even
biological and chemical or fluid mechanics. All that is needs to do is
map states of the incoming tape to periods of time of some state of the
actuator which moves a physical thing. There are probably in practice
many ways to do this. It could certainly be done using a primitive copy
of existing technology, based on silicon and deliberately added
impurities. But that is probably not the most effective path. We have
learned a \emph{lot} about organic electronic devices, biological and
chemical systems, and even mechanical design in the last hundred years
and we have no idea how many simple solutions to our problem will
present themselves until we try.

Always, our goal is to keep in mind the basic notion of turning physical
media on the incoming glyph tape into geometric actions. There is
already significant precedent for doing this without modern computers in
the automation systems of the early 20th century. But again, what we are
doing is much easier than what they did because we no longer need the
extremely large scale, we are only trying to build systems which move
2-5 motors relatively slowly doing simple things for craft-based
production.

Once we can make our own circuits from scratch using found trash
materials which control motor motion, we immediately make sure this can
be used to print the physical media which forms the incoming Geometron
glyph tape. This basic system, where we replace the Arduino with our own
circuits, use those to make fabrication robots which print the code
which prints the circuits which makes the robots and so on, we have
started to fully close the media loop. If this is done using prints on
clay, and those prints are used to fabricate human readable media,
machine readable media, and circuits, that closes all the loops. We then
want to move from that up to screens, by building electronic circuits
the sole purpose of which is to control what pixels light on on a screen
when, again with purely geometric programming. This is the final step in
full Geometron. When we can control old trashed screens from the
existing system using circuits we can make ourselves from scratch using
robots we make with that same technology we have a fully
self-replicating trash based media which requires no mines or global
supply chains. If this media carries all the code to control all the
machines to build more media we have full self-replication.

There are many steps left out here. Not all media will be discrete
geometric actions. We also need to be able to display bitmaps and play
sound. This means we need physical media which directly stores those,
which will mean going back to more analog media using our physical media
fabrication. Digital media as an alternative to analog media is
\emph{primarily} a tool of domination and control. When media is part of
a giant system of arithmetic, the only added ``value'' of that is that
it makes it possible to impose controls on what can or cannot be played.
Returning to analog, and then combining that analog with discrete
geometric controls of where it will get displayed as well as direct
human controls seizes power back from the people who have used computers
to dictate and control our lives. The machines we are building here are
not ``computers''. They perform the same functions as computers but we
use different metaphors to describe them and that totally changes how we
relate to them. Removing computers from our lives is a political act,
and we cannot do it soon enough!

This fully self-replicating trash-based media is then used to carry the
evolution of our other systems, documenting the research and development
process as we approach our final goal of building everything to provide
all human needs for free for all people everywhere from the sun, trash,
water, and the living Earth. This might sound outlandish, but as we hope
we have shown, it is not. Each step in this path is something
straightforward which can be done using simple experimentation with
already-working technology. No miraculous new technology is required
here, like molecular nanotechnology or strong AI. This is just a choice
to build a new culture and a new society based on a shared value set. It
is not really new technology, and that is why we know it will work.
