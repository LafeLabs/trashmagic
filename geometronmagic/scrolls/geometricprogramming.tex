
All technology is based on geometry. When a screen displays text or
graphics, that is a geometric construction. When we build integrated
circuits which form the basis of modern computers, that is just a
complex pattern of shapes on a flat surface, yet another geometric
construction, as are the circuit boards on which they sit. Architecture
is all geometric construction. The paths of cutting tools or 3d printer
nozzles through space are geometric constructions. The path of a tractor
or harvester going across a field in a sequence of rows is a geometric
construction.

Any manufacturing process can be understood as a geometric construction.
To build the full Trash Magic we aim for in this work we must create
manufacturing processes which replicate from place to place freely over
communication networks. If all manufacturing is a geometric process, the
most fundamental way to transmit manufacturing processes is to build a
language on geometry. This represents a shift in thinking from the
dominant machine ideology of today, that of the computer. In the current
dominant ideology, the most fundamental things are numbers and
arithmetic operations. People think of computers as being engines which
use numbers to decide what to do to other numbers. All of the various
automation and media functions are considered to be ``applications'' of
this arithmetic engine model.

In Geometron we re-imagine the fundamental idea of machines as always
being for doing geometry, and controlled by geometry. This is a
philosophical shift, one of values. We choose to value the symbols
displayed on a screen and the path taken by a tool for manufacturing as
more fundamental than ones and zeros in the arithmetic model.

In computer theory, people have an abstract model they use to describe
all computers based on long tapes of ones and zeros which control the
movement of the tapes and the operation of doing things with numbers.
Computers are then judged in terms of how effectively they process
numbers, in a very numerical way, where processing more and bigger
numbers faster is a measure of power. This has led to architectures
which use an absurd amount of computing power, with machines that do
arithmetic operations a billion times a second all to carry out some
task a thousand times a second, or even every few minutes. By
considering the geometric actions to be more fundamental we aim to move
to much simpler machine architectures which will allow us to build
machines with less intense technology than modern microelectronic
fabrication methods.

We aim to close all the loops in machine fabrication, using a geometric
programming language to design and fabricate simple circuits which can
run the geometric programs which make more circuits and so on. Rather
than trying immediately to build the most powerful machines, we aim to
get as quickly as possible to machines which are able to run the entire
process of replicating them using their own technology, without any
external input. The details of this process will be discussed in another
chapter, but it is an essential part of the whole Geometron/Trash Magic
process to build these loops of self-replicating physical media.

Just as computer scientists create toy models of imaginary machines
called Turing Machines(after mathematician Allan Turing of WWII
cryptology fame) which act on numbers, we create a generic toy model for
how to create geometric virtual machines. This is called the Geometron
Virtual Machine.

In the Geometron Virtual Machine we imagine a tape of addresses, much
like the ones and zeros in the Turing Machine. These addresses represent
positions in a pair of imaginary cubes in space, together called the
Geometron Hypercube. To make things completely geometric, we imagine the
physical tape as having a sequence of symbols made up of arrays of dots,
where each pattern of dots represents an address in one of the cubes.
Each address in the cube itself contains a tape made up of addresses. So
this makes things able to endlessly refer to themselves, since the main
tape can reference an address which references another address and so
on, building up whole complex networks of geometric actions. Everything
is recursive.

The virtual machine can also do physical things based on each address.
This is where it is a different model than the Turing Machine. The
Turing Machine can be \emph{used} to control physical machines but in
its basic model it only works with ``pure'' numbers. The Geometron
Virtual machine has physical operations built into the definition of its
structure. Different areas of the address space do different kinds of
operation. The details of this structure are covered in the First Book
of Geometron. The most basic operation in our system is display of
graphics on a screen. This is because the display of symbols is how the
machine interacts with the human mind, and again this points to how this
differs from the Turing model.

The Turing model ignores the human operator. It imagines an infinite
tape which can in theory run programs forever, ignoring the humans who
operate it and the physical mater it interacts with. In Geometron the
human has multiple roles. It starts with a human pushing some kind of
``start'' button which starts the main tape being read, and we generally
assume the tape is finite and that it only runs one way, once(as opposed
to infinite, and running forever). Each address on the tape has a
corresponding symbol which is in a human readable format, where by
``readable'' we mean that it has some clear meaning to the human
operator. These symbols are themselves constructions of Geometron. We
aim for these symbols to \emph{be} the language used for programming the
machines which make all our technology. Different buttons can trigger
different different glyphs, which is how we create direct machine
controls like ``move left'' or ``move up'' for robotics.

Our lexicon of human readable symbols we put in the Geometron Hypercube
includes the entirety of whatever written language a human operator is
normally using, such as English. For instance, for English we have
addresses in the Hypercube for each of the printable letters on the
English computer keyboard(which are the same numbers as the ASCII code),
and each of these represents a sequence of geometric actions which taken
together draw that character. In the current software, this means
actions taken in a web browser which control where pixels are in a
graphic, a physical construction on a computer screen using the
programming methods of the browser. This is called a font, just as in
other computer software. A big part of what makes these methods powerful
is how they can be mixed and matched with different physical means of
geometric action. For instance, a font can be constructed out of
discrete movements and drawing of pixels which can be cloned from a
robot which impresses dots into clay with a nail to a bitmap on a
computer screen to spray paint dots on a wall to microscopic laser burns
with no change in the code.

This ability to move geometric construction from one physical
fabrication machine to another goes to the heart of why Geometron is a
critical enabling idea for Trash Magic. When we build our fabrication
technology from parts of machines we find in the trash, having this
geometric description of what the machine does independent of the
details of the machine makes it much easier to adapt programs from one
machine to the next as our system evolves. A program to make a square
with a tool which consists of ``start drawing, move left, move down,
move right, move up'' can be done at any size with any tool once we
build a virtual machine model which maps those movements to what the
motors do.

Of course this is all still based on computer programming languages on
machines that do arithmetic. But the purpose of computer programming
languages is not just to control machines but to make it easier for our
minds to think about how to control them. This is where Geoemtron really
does things Turing machines don't, by integrating the process of
designing languages, controlling automation, controlling fabrication,
and building abstract language structures.

We \emph{choose} to consider geometry to be more fundamental than
arithmetic. We believe that this choice is not just a mathematical one
but a moral and philosophical one. We believe it represents a shift from
an information economy based on replication instead of production,
communication instead of domination and control. We use the slogan ``no
war but the math war'' to represent this idea, that we believe that the
ideology of numbers is integrated into the ideology of permanent war
that dominates the world today, and that shifting away from that way of
thinking requires this change in mathematical philosophy. The war
machine of today is needed to project power over the long distances
required to keep the supply chains flowing which keep material moving
from the mines to the consumers. Free replication of geometric
constructions in locally sourced trash represents a shift away from
empires of control and towards sharing and abundance.
