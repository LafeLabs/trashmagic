

Cybermagic refers to sets of computer files which include scripts to
replicate the whole set as well as both documents on \emph{how} to
replicate the set and also on \emph{why} we want to replicate it. We use
the term ``magic'' here to mean sets of things which include the
\emph{desire} to replicate the set. The files and hardware themselves
never warrant the term ``magic''. We apply that term only to refer to
the property which makes people actually have the desire to choose to
replicate the set. It is this human intention which animates technology,
and that is what we call ``magic''.

Cybermagic is self-replicating code which can all be replicated freely
from one Geometron server to the next entirely from the Web Browser. We
do not ``install'' software. We use only code which can be run from a
browser without ever logging into a server. To make a self-replicating
set of computer programs we have some scripts which copy all the other
ones, some files which load, save and delete files, and some which
catalog them. This set of files can combine with any other set of files,
and together can build self-replicating sets of files, where the
\emph{entire} set is managed from a Web Browser over the network without
ever logging into the server.

This chapter will get a little technical. It is to explain to people who
know some things about computer programs how the software here is
structured. For those with limited technical knowledge but some
interest, we try to describe all the terms in the hope that this can be
an invitation to learn all these languages and become a Geometron
developer. Email the author with any questions.

Our intent is not to recruit developers into the project of co-creating
this system but to teach people from scratch how to work on it, to build
a whole new culture of creating software with no link to the existing
one based on profit and control. This is a political and social choice.
We believe that the work of professional software developers who work
either for money or for free but in support of commercial software do
great harm with their work and that making a hard cultural break with
this group of people is necessary to build system which have more decent
human values than those that dominate our world today. For this reason,
this chapter has to both be a little bit technical to try to invite
people to learn and join us, but not technical at a level which assumes
someone is already a developer as it is our intention to avoid working
with anyone who involved with the software industry at all.

The main formats of files we copy are HTML, PHP, JavaScript, JSON, and
SVG. HTML, or HyperText Markup Language, is the primary language of all
content which displays in a web browser. It is the language made up of
``tags'' which are words or letters between angle brackets along with
the raw text that makes up a web document. The language used to talk
about HTML is very clearly inspired already by the ideas of set theory,
as the word ``element'' is used to describe all of the kinds of things
which exist in a document, like paragraphs, images, links and so on.
HyperText refers to the way that documents can all link to each other,
making the entire Web in some sense one giant document where documents
all link to each other. It is important to note that the ``web'', based
on HTML is not the same as the Internet. An HTML document can exist on a
machine not connected to the Internet and much of what exists on the
Internet(which is just the network of physical devices) is other kinds
of traffic like phone calls, emails, and other data. While the Internet
was a creation of the US military back in the late 60s, the World Wide
Web, browsers and HTML were all created at the European particle physics
lab CERN at the end of the 1980s as more of an academic project. By
default, the file index.html is the one loaded as the home page on any
given web address. So if you point your browser to a domain name without
a file name it just displays this file. We always need an index.html
file to exist and replicate for the system to run smoothly.

JavaScript is the language which is part of the HTML standard which does
actions, like making buttons or text inputs work, calculating things, or
manipulating the HTML content on a page. Whenever possible, our first
choice in this system is to use JavaScript for all code that does things
because that can exist either in an HTML file(in the ``script'' tag) or
called directly from an HTML file.

The only language we use to interact with files on the servers is PHP.
PHP is a old language by web standards(1995). PHP originally stood for
Personal Home Page, but it now stands for the recursive initialism PHP:
Hypertext Preprocessor. It is a language specifically designed for the
task we need: doing things on a web server entirely from inside a web
browser over the network. The first and only thing we need to do in
order to install Geometron is to copy the program replicator.php onto a
web server and run it. That's all! This script calls a file called
dna.txt which lists all the other files, and the program uses that list
to copy every file in the set. So taken together, replicator.php,
dna.txt and all the other files on the server are a self-replicating set
of programs all of which replicate when anyone on the network puts
``replicator.php'' into the browser. This is what makes it incredibly
easy, fast, and free to replicate whole sets of documents across our
network: it's just links you can click on. All the code in the set is
edited using another PHP program called editor.php. This program uses
the JavaScript library Ace.js to add syntax highlighting, and loads and
saves files using helper programs fileloader.php and filesaver.php.

PHP files are all stored in a directory called ``php'', and use the file
extension .txt so that they can be read in a browser without running
them. A program called ``text2php.php'' finds every single .txt file in
the directory php and copies them into a .php file in the main web
directory. This can be thought of like ``compiling'' the program,
although it really is just copying the files, and not doing anything
else to change anything. The dna.txt file is generated using yet another
program called dnagenerator.php.

Data like the list of files to replicate are in the JSON format. JSON
stands for JavaScript Object Notation, and is another language clearly
inspired by set theory and foundational mathematical ideas. JSON is a
minimalist way to organize information into either arrays of pieces of
information like text or numbers or name-value pairs which have a name
which is just text and a value which can be any of a number of types of
information. All these can be fractal, with objects inside arrays inside
objects and so on. This format is used for a whole range of different
Geometron applications to store data.

All the icons used in the system are in the vector graphics format
SVG(Scalable Vector Graphics). These are also part of the
self-replicating sets. All of these icons are created from scratch using
the Geometron geometric programming language, again all from inside the
browser over the network. This shows how self-contained this system is.
Graphics, scripts, format, content are all things we can create,
organize, edit, delete and replicate again and again entirely from
within the web browser over the network. SVG files created with
Geometron have JSON embedded in them which contains the Geomtron glyphs
used to create the symbol, as well as parts of the Geoemtron Hypercube
which are referenced and the style information which specifies how the
file is formatted.

PHP programs can take inputs using the text you put into the address bar
in the web browser using question marks and ampersands. This allows
people to get a huge range of control over the system from the browser,
creating new files, destroying old ones, forking the system into new
directions on any given server. We use this for instance with a program
called copy.php to copy files from anywhere on the network to anywhere
on the server we are interacting with. This is also used to create a new
file using the editor. For instance we can create a new html file called
new.html simply by putting into the browser address bar
``editor.php?newfile=newfile.html''. Then we can edit this new file,
click on the link from the editor to create an updated dna.txt file, and
the set which gets replicated by replicator.php will now include the new
file. Using PHP programs called from the browser address bar can replace
command line operations which we have banished from our system.

This is anarchist software architecture, one of constant chaos. There
are no restrictions. Any code can run anywhere any time by anyone. Any
file can be deleted by anyone at any time with no log in, no password,
just a click and you destroy anything. The same is true for replication.
Anyone can copy anything to anywhere at any time. The only restrictions
on what is ``private'' or ``public'' servers are based on physical
network parameters. Local networks can have servers which are not
visible outside the the network. This allows for networks of servers to
exist in a shared public space, with constant local replication, as well
as replication \emph{from} all globally available servers, but without
anyone outside the local network able to interact with the servers. One
can think of a server only on a local network as having a one way valve
for information from the global Internet to the local server. We can
build communities of constant co-replication and co-creation of
documents over a local network. It is wifi anarcho-communism: a wifi
network which abolishes the concept of property, the concept of the
``user'', all private data, all private code, all private documents, all
restrictions on user actions, and indeed the concept of the ``self''
itself. This is a universe of files without property and without
individual identity.

A stripped down set of the absolute minimum collection of files can be
useful for understanding the structure. This consists of just the home
page index.html, the replicator, editor, dnagenerator, txt2php,
filesaver, fileloader, and dna.txt. That set of files behaves like a
living thing. It can replicate, evolve, and replicate again. If it is
part of an already functioning Geoemtron system, it can also be
destroyed by destroying the whole branch it exists on. With servers
already in place and human operators already maintaining the system,
these sets are like organisms in an open ecosystem we all maintain
together. We are not ``engineers'' who create static technology, but
shepherds who nurture and grow a living system.

The sets of files we replicate can include any of a few different file
types, each of which has their own self-replicating infrastructure to
support it. This includes the image set, which is a set of images people
can upload to a server, delete from the server, and replicate as a set.
There is also a symbol set replicator which includes the whole Geometron
system for creating and editing graphics and saving them to .svg and
.png files. There is a generic file set replicator which has not
specific format specified. This is useful for files like CAD layout
files for circuits or programs other than those in the cybermagic
system. One example of this kind of file we will use a lot is the
Jupyter notebook, which is a very useful tool for all kinds of science
and math calculations and education, already widely used in many fields.
As with all elements of the Geometron Magic system, we are creating sets
of things which include us, the creators of the sets, and which
replicate themselves(with our participation). When these things are
computer files, that's cybermagic. With some simple copy/paste it is
possible to expand this to any type of sets of files, growing our system
to do a wide range of things.

The Map Books are combinations of scrolls and maps which can create
swarms of documents which are graphical and text all linked with each
other and all of which replicate together. The Map Book can form the
basis of physical hypertext documents, which are documents combining
maps of physical spaces with hyperlinks to text documents relating to
those spaces which can link back to other physical spaces and documents,
and so on. Physical places can have physical media pointing to a domain
which hosts copies of map books which have links to documents which act
to change that space by guiding people to alter it themselves along with
the documents. This creates feedback loops of physical media in physical
spaces, mediated by our digital media, which can be a powerful
transformative force. This mixed reality media can also form the basis
of complex games of many kinds, the structure of them is left to your
imagination. The Map Book is central to our system! It activates
physical spaces and creates new worlds for us to live in. But it is also
just a self-replicating set of files like everything else here.

Any set exists at some point on a server and that can fork down to sets
in directories, which can fork again and again, making more sets which
can be Magic Books, sets of code, sets of symbols, images, or any other
file or document. All this happens with a page called fork.html, which
allows us to create forks of whatever name we want and replicate the
book down a level, and the book can be replaced with any other set using
another replicator. Any fork can be deleted along with all its sub-fork
instantly at any time by anyone. Everything is fractal.

The only thing that preserves information in our system is constant
replication, just like life, which constantly reproduces in the face of
constant death. This is living media. We walk the Earth in the physical
world carrying our web servers and our physical media and constantly
replicate swarms of code and documents from person to person in our
physical space. And we remember that the magic is not in the code or
machines but in that spark that jumps from one person to the next when
we are able to project our desire to build this free network into the
minds and hearts of new people. This is cybermagic. Code which carries
the media which replicates the desire to replicate the code which
replicates the media which describes how to replicate the code, all on
physical infrastructure which replicates with us, the People of the
Network.

To replicate this system, see the installation instructions in the
pibrary Github repository at www.github.com/LafeLabs/pibrary.
