
This is a book of direct action. This is not an appeal to authority or
an agenda or policy plan for existing organizations, but a framework for
directly choosing to go build what we need to build. Some of the things
we need to build will take millions of people decades to build, and will
require vast organizations which we can scarcely even conceive of today.

All of this starts with simple direct actions using only the resources
we have available to us right now. In this chapter we go through the
things we are asking as the creators of this work in order to grow this
new economic system to the scale needed to solve global problems. This
requires a delicate balance between finding the \emph{right} people with
specialized skills who can build complex machines and dedicate their
lives to creating new social structures and spreading our message and
philosophy to a broad enough audience to provide community support for
our core of creators. We begin with the small steps that everyone can
take. This is magic, as defined in this work: the replication of desire
to build this world, nothing more and nothing less. Without this,
nothing else will work. And if we get enough people to believe, we can
do anything.

To spread these ideas, this book itself must replicate as broadly as
possible. This means we want the free digital form in as many formats
and languages as possible on as many web pages and computers as
possible, shared with as many people as possible. We also want the
physical bound copies from a professional printer to be distributed as
widely as possible. This is where the author's self-interest must be
publicly declared: the most straight forward was for me, the author, to
live off of this network is for people to buy so many of the physical
bound book for dollars that I can live off it. If I can live off of book
sales, I can work on more books without any strings attached, and this
will maximize my ability to push all this work to the next level without
distractions. Physical copies need not be purchased, they can be printed
and bound yourself, by printing the letter size pdf and binding with a
simple loose leaf or spiral binder. Purchasing books is both to get
higher quality bound volumes in a smaller format and to support the
author. Also you can be Trash Robot. And then we are Trash Robot. Trash
Robot is an art collective, and anyone can print and sell books without
any permission or payments. Just ask us how.

We are looking to saturate certain physical spaces with these ideas in
order to activate those spaces into our network. To do that, we ask
people to buy or print as many copies as they feel they can and
distribute them for free in public spaces where we aim to spread the
network. These books are not property. In addition to being free of
copyright since they are all published on a Public Domain license, we
are asking you to never claim the physical volumes as property but to
release them into the commons as well. All of this is tied to the other
physical elements of the network. Our network is built into the physical
street, with physical media and computer and network resources all
focused along some area. We look in that area for anyplace that can
naturally hold public books and drop books there. This includes
libraries, book stores, coffee shops, waiting areas of medical offices,
lobbies of apartment buildings, art galleries, schools, religious
institutions, break rooms in work places, infoshops, ``little free
libraries'', hotel rooms, and community houses. All of these are
strategic placements. We are always using the placement of the book not
to simply find readers at random but as a campaign to activate a
specific place, to build a new emergent network of people in that space
who share our common purpose. We aim to place many copies within a
couple mile radius of a location, rather than broadly distribute, and
always the distribution is a means to an end, where we aim to expand the
other physical elements of the network(media and machines).

Beyond the message carried in the book, the next thing we are asking
from people who wish to replicate these ideas is the replication of the
other physical media, which point to the digital media. The simplest
physical media is the patterns we can create using Action Geometry as a
ritual artistic practice. The simple geometric shapes described in the
Action Geometry chapter can be created for free paper, pen and
scissors(or even careful folding and tearing). These can be used to
create a whole universe of tiled patterns which can be used to spread
the idea of self-replicating geometry. These are very recognizable, and
create a sort of brand identity which can be on any media. This media
can include cardboard signs held by people on the side of the street to
wall murals, tattoos, chalk art on sidewalks, sewn patterns on clothing,
Easter eggs on circuit boards and microchips, computer games, game
boards, and really any new type of media we can think of. Any pattern
created with this system can be easily replicated by anyone using the
same system. This transmits not just symbols but the \emph{idea} of
using geometry to make self-replicating symbols. That idea is the most
important part of this, since it is the ``magic'' which drives the whole
system, the replication of the desire to build this world of freely
replicating things.

As with the books, these geometric patterns which can freely replicate
will have the most impact on the growth of our network when they are
displayed strategically in the physical locations we wish to activate
into the network. These physical media elements point to our digital
media which describe the system, including the digital version of this
book. This can be as direct and physical as a cardboard sign next to a
Raspberry Pi or it can be a sign pointing to a web address which hosts
our collection of Magic Books which spreads the network.

Fashion can be one of the most powerful forms of self-replicating media.
Shirts, pants, robes, dresses, skirts, cloaks, hats and accessories can
all carry self-replicating geometric arrangements of shapes, where each
instance is constructed using the shape set and construction methods of
Action Geometry, and where the whole is used as a brand identity to
spread the core message of our network. If we use clothes and other
cloth products which are discarded or donated, this further promotes our
message through direct action: the article of clothing both embodies an
ideal and transmits a geometric pattern which represents that ideal.
Furthermore, developing a tradition of creating extremely recognizable
fashion based on our ideas and methods creates a very obvious identity
in which we can recognize each other and be recognized in public,
creating a more freely replicating culture. This fashion culture also
plants the seed for the full Trash Magic textile production we will
produce in the Trash Factories when we start to establish more
substantial industrial infrastructure in our system like mills and looms
which create cloth from plastic bottles and industrial sewing machines
creating patchwork trash cloth for use in new clothing and shelter
products. All of this clothing production, like everything we produce,
is a hybrid of mutual aid directed at the most marginalized people and
commercial products sold to support our ongoing operations.

All this can start simple. Just a t-shirt with a couple stitched colored
triangles with some obvious symmetry can require minimal skill and
effort and no money to make, and can be enough to start a conversation
which leads to spreading these ideas.

The next replication action we ask of the reader is participation in the
symbolic economy of Icon Magic. This means designing, creating, and
replicating the small pieces of physical media which carry the simple
pixelated designs of icons drawn as Geometron glyphs which are described
in the Icon Magic chapter. For these to freely replicate, they must be
distributed and used in a way where the clay prints and stamps are
carried with their finished products, so that more can always be made
with more clay. The ``factory'' to produce more of these consists of
just a block of polymer clay, access to a conventional oven, paint pens,
and sand paper. There are numerous other ways to replicate them,
stamping symbols into heated plastic, or casting pourable materials like
epoxy resin, silicone or chocolate. As long as the original prints are
available to make stamps, we have a system which can replicate an
ever-increasing flow of media.

These Icons can be used for many things. In all cases, however, as with
all the other media presented here, their primary purpose initially is
simply to replicate the desire to replicate the system. So we design
them based on what we think people will care about for \emph{anything}
they might already care about. They can be board game pieces, markers in
public spaces, white rabbits which are to be found and direct people to
online resources, used like cards or rune tablets for ceremonial
purposes, used for all sorts of rituals, turned into jewelry and
worn(earrings, cuff links, belt buckles, pendants, buttons), used as
barter tokens for goods and services. As discussed in the Icon Magic
chapter any ``thing'' in the most abstract sense be it an idea, object,
person, place, action or set of other things, can be represented by
these pieces of self-replicating media.

Participation in this economy starts with simply accepting the physical
media from someone who already has them and passing them along. As with
all our media, this is not property, it is intended to flow as fast as
possible from person to person to replicate all our ideas and culture.
Simply take and give, carry and share.

The next level of complexity from simply sharing the Icon Magic media is
replicating them yourself. This means learning the clay craft, which you
can learn from someone who already knows how to do it, and repeating it
yourself. Even easier than learning the clay craft is learning to design
your own icons using the Geometron software on any given Geometron
server. This can be done on any Raspberry Pi server, and the only
product is a sequence of numbers which forms the code for the Geometron
glyph which is printed out by the printer robots made from trash. This
is just a text string, and can be copy/pasted via text, email, or direct
replication from server to server across our network. Creating these
icons can also be done by free commission, where you think of a symbol
and ask someone who already knows the system to make it. This simple act
of thinking of a symbol you want is one of the most valuable in the
growth of our system, as it is the signal which determines how the
system focuses on what people care about. This act is as simple as
asking a robot operator to create a symbol.

While these pieces of media have many purposes, we must always remember
that our objective of making everything free for everyone requires that
we always provide resources to those who have the greatest need in any
given community. This means that directly selling all these
self-replicating media can and should be used as a way for people with
nothing to support themselves. If people on the street who need money
can sell media which they can replicate indefinitely, they create a sort
of mutual-aid based currency, in which the thing they sell to someone
with more money and resources then represents the information that
someone was helped, which can be transmitted through the rest of our
social network. Initially this looks a lot like a currency, like money,
even though it is \emph{not} money. It is not money because it can be
freely replicated by anyone, and because each icon \emph{means}
something. Money is designed to \emph{not} replicate freely, and to
represent only number. This is designed to represent a ``thing'' in an
abstract sense which is \emph{not} number, is an expression of a
Geometron glyph which is pure information, and can replicate and evolve
forever.

We can make practical products this way which people would normally pay
for, like attractive and interesting jewelry and completed board game
sets like self-replicating chess sets. This is a hybrid between viral
media, craft production for profit, currency creation, and the
generalized philosophical language of Icon Magic.

All of this Icon Magic media creation of course relies on physical
machines to make the original prints. This is done with the Trash Robot
Geometron printer, made from old DVD drives, plastic trash, cardboard
trash, duct tape, Arduino, and some simple off the shelf electronics.
All together, these robots cost about 50 dollars in parts, and can be
assembled in a day. The skills required are soldering, very basic
electronics, and carefully cutting and taping trash into shapes(safe use
of a box cutter). We will spread the creation of these robots the same
way as all the other parts of the system, where people who know how to
build robots share with others. If we can make the products of the
robots worth something(both in money and in non-money value) that makes
the robots worth something. If the robots have value and area easy to
copy, they will replicate. Once we get robots made out of trash
replicating freely, we have the basis for building all the other things
in our system. This robot architecture can then be evolved into machines
that make other machines, machines that make smaller and larger and more
complex physical media(like printing books on plastic), agricultural
automation, and all the other elements we need to build full Trash Magic
and full Geometron as described in other chapters of this work.

Each physically local network hub requires that we build mixed reality
media into a public space. This is done with a combination of the
physical media here and the digital media of the Magic Books and the
infrastructure of the Pibrary. Part of this is building the Map Book of
a locality, which means writing text documents and creating maps which
integrate stories and knowledge and links into the physical landscape.
This only requires that one or two people in any locality really
understand our software. The main labor is in the mental process of
giving meaning to a space and sharing that meaning. For this we need
story tellers, people who are good at connecting people, people who own
or control shared spaces, people who spend their days occupying public
spaces, and really anyone with an interest in this project to think of
types of information they want embedded in a space and to talk with
someone who can use the software to integrate all that media into the
space. An Operator who knows the system then compiles all these
documents in a self-replicating form and replicates them to public
facing web pages which are pointed to via the physical media in a
physical space.

This system is part of what makes our network financially viable.
Building the so-called Map Book or Street Book of a place can link
people in a place, and that linking can create enormous economic
benefit, which can be kept in the local community and used to materially
support our operations. In the simplest sense, this just means we
advertise all the local small businesses in a given area and ask them in
return to host our infrastructure for free and provide us with resources
like free food and a place to work and live.

The aforementioned Operator is someone who learned directly from an
existing Operator how to create, edit, delete, and replicate all the
documents in our system. We invite you to learn this by asking an
existing Operator. You will learn how to create text documents in this
system, how to work with our maps, and how all the replication of
Cybermagic works. You will learn the ins and outs of the Raspberry Pi,
enough to teach other people in the community to learn to use them, and
will share them in public spaces with whoever has the greatest need for
that resource.

In addition to the Raspberry Pi operators who know the software and
basics of the Pi hardware, we need to train network operators who can
build and maintain wireless networks which project free wifi into public
spaces. This means learning what to buy and how to install it for
wireless point to point links and hotspots, as well as learning the
logistics of setting up a dedicated Internet connection in a convenient
location which is the source of the network connection. We are not an
ISP, and are not selling access. We are simply training people on the
level of one small space to create mutual aid based public wireless hot
spots.

All of our documents are also always replicated to public-facing web
pages on domain names which are connected to physical locations, and we
need to ask readers to help build these as well. This means buying a
domain and then paying for the monthly hosting costs. In general we
assume this will be done by the same people who are maintaining the
Raspberry Pi infrastructure, as this is much easier than that, and is
just a mirror of that. A domain can cost as little as 10 dollars and
hosting can be as little as 10 dollars a month, an insignificant cost
for a network which provides any significant value to commerce in a
local area.

The Cybermagic code which supports all this also needs developers who
can learn how the code works, edit it, and replicate it out to the world
using public open source code repositories. As with all other elements
of the system, we invite you to contact an existing developer to learn
the system and replicate it. The only skills you need to learn this are
basic HTML and JavaScript along with a very basic understanding of what
PHP does. This can be learned from a combination of an existing
Geometron developer and online free resources like w3schools.com and
codepen.io.

This network is physical. So we need to physically integrate into a
space in order to activate it into the network, and that means we need
to travel and stay in places for days or weeks to fully replicate the
system. We are also asking readers to invite us, the creators of this
network to stay at your home and work in your public spaces to build
this. If you can feed and house one of us for a week or two, we can
build out all this for free.

In order to build significant hardware infrastructure like long range
solar powered wireless Internet links in rural areas, it will be useful
to have funding which is not tied to commerce, but is simply there to
build things. To this end we need to get grants for network expansion.
These grants can come from any kind of sponsor who supports addressing
``digital divide'' issues. Both physical access to
computing/communication resources and the skill set to make use of those
resources are one of the forces driving inequality in today's world.
Many governments and non profits recognize this issue and have earmarked
significant funds to address both of these.

Our system constructs free network access as well as free computers in
public spaces targeting the most marginalized(starting with homeless and
travelers) and trains the local community to use them. This directly and
in a very cost effective way addresses precisely the agendas of these
sponsors. Furthermore, each time we replicate the system, we are
training people who are then qualified to further apply for grants to
support further expansion. Grants we are aiming for are to build things
in a specific space. They pay for our time for some period of training
and building and also all the hardware. We are aiming for grants in the
range of 25-250 thousand dollars over from 2 weeks to a year. Grants
will be applied for in collaboration with existing non profits,
preferably the public library or university. Grants can be from national
governments, tribal governments, large non-profits, or just donations
from high net worth individuals or corporations. We are asking readers
who are grant writers, workers at a local non-profit, university
faculty, librarians or local government people to collaborate to write
these grants, and then to pay us, the creators of the network to live in
your community and replicate all this. We are also asking readers who
fall into the sponsor category as government officials, high net worth
individuals, or people at NGOs to help find the right grant recipients
to put these collaborations together.

The content of these grants is to buy Raspberry Pis, buy solar power
stations, buy wireless network hardware, buy domains, and buy parts for
the robots and art supplies, and then to train people in a local
community to build physical social media which self-supports in the
community. Most grants meant to address the digital divide are not
sustainable for communities since they train people to leave to get high
paying technology jobs in cities. Since this provides no direct benefit
to communities, this means there always needs to be more grant money to
sustain the program. We provide a completely different model which is
much more cost effective and sustainable, in which the product of our
efforts is a \emph{locally controlled} network with deep local knowledge
which supports local commerce. This makes the network financially
self-sustaining without need for more grants, and keeps the technical
skills we teach local, preventing the brain drain that makes a lot of
workforce development programs in rural or economically depressed areas
self-sabotaging.

For now, the main things we are asking from our readers are what are
listed above, which is just spreading the network and the idea of the
network. As the scale of our network grows, we are asking for people to
create more and more complex new things. We want people to write books!
We need you to write all the books we need to build full Trash Magic. If
the network is self-supporting, this means you can make a living doing
this! Once our network grows, more and more people should be able to
quit working for the consumer economy and move to working full time for
the network off of the donations and revenue available from those
getting benefit from it. When this becomes viable, we will be supported
by the shared desire of the community and can focus entirely on the most
important problems for building full Trash Magic. We need you to tell us
how to grow food everywhere, how to fabricate all medicines we need, how
to build heat engines, harness flowing water, work with the soil to use
it and improve on it and live in it, to tell our stories of our shared
culture, to create art, and to just form the social matrix that connects
all this together.

The bigger we get and the more of these problems we can turn into
self-replicating media, the easier all of this will get for all of us.
By directing all the benefit initially to the \emph{most} marginalized
people in the most public spaces, we grow those spaces and build a new
civilization where the baseline of life is comfortable. We want to
abolish poverty not by moving everyone into a big house with a nice car
but by making the tent cities in public spaces into places of luxury and
abundance, with free medicine, free air conditioning, free food, free
clean water and sanitation and so on. This is our only path to freedom.
We must totally abolish the want of those who have the most want
\emph{first}. When we do this, the vast store of human energy which is
currently ``not working'' can form a social network where everyone has
value, where just sharing in the community itself is considered a thing
of value because it is part of how we replicate our whole civilization.

All of this is much closer than you think! We can do it. It just starts
with spreading this message, it starts with this book, please share it
and we can build all this! As you learn things, teach them to others. As
you build things, share them. And please help us as you do this by
creating media and sharing it, creating videos and other social media
posts promoting all these ideas. Every time someone shares all this, the
whole network gets more powerful and we get closer to building our new
world.
