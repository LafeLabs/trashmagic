

Our aim is to share technology which makes all of the elements of a good
life free for all people everywhere. The technology we need to do this
exists. From cheap local renewable energy to dense growth of food to
safe disposal of waste, we have all the material elements of a life of
plenty created by our shared knowledge. And yet we live lives constantly
hounded by scarcity based on activities which are in the process of
killing us all. Why? What is missing from our collective lives? This
work is an attempt to answer these questions and to provide a path
forward using new ways of engaging with existing technology to build the
social structures needed to get from a path of destruction and scarcity
to one of creation and plenty.

Our current model of how we think of technology and ``the economy'' is
based on production and consumption. In the modern world, material is
extracted from the Earth, is processed into ``products'', which
eventually turn into waste, and then the process repeats. This process
will always produce scarcity, as everyone must compete for the limited
resources. That scarcity is managed by people claiming ownership over
the land from which resources are extracted, ownership of the machines
which produce products, and control of the workers making the things.

We cannot continue with this model because in a finite world it will
always consume all resources and destroy all life. We should not
continue because it inflicts misery and fear on all but the small group
of people who own and/or control the system.

Furthermore, even things which are not based on this model, like writing
books or software which can freely replicate on existing hardware, are
forced to conform to the basic logic of this model. While money no
longer literally represents metal dug up from the ground, the
\emph{metaphor} of money is still based on that. No matter what anyone
does, we all need money in our existing system to survive. Under the
existing system if a group of people with no money all want to build a
thing, they can't do it without someone with money-creation authority
blessing the activity. People can physically do it in theory, but as
long as the material needs of survival are controlled by people who
demand money for those things no one but the rich can afford the luxury
of doing useful things for which there is no monetary incentive. This
means that the more people in society produce things which \emph{don't}
require digging things up or doing manual labor, the larger the gap will
be between the money production process and actual creation of value.
This is the reason inequality will always get worse as the information
economy grows. Everyone in the information economy is making money from
replication, while the old economy still runs on production. The more
powerful information technology becomes, the larger this gap will
become, and the rate of increase will keep accelerating.

What we need to recognize in order to move beyond this system is that we
must transition from a consumer society to a replication society, and to
change our value system to reflect that. In the last 300 years we have
dug up a staggering quantity and diversity of material. None of what we
have dug up to build our shared global industrial civilization in the
last few centuries is really ``thrown away''. It's all still here, and
generally way more of it than we want. Some of it is in landfills, some
of it is in wasteful machines that have no reason to exist other than to
keep ``the economy'' going, and some of it is in poisoned water and
soil, but all of it is still around us in some way.

The laws of physics and chemistry will allow us to re-use all these
elements indefinitely just as natural ecosystems have with simpler
elements for the duration of life on Earth. We can in some sense think
of all the trash, toxic waste and useless junk we have created as the
``hardware'' on which we want to create the ``software'' of a just and
sustainable future. In this model, value comes from the power of
replication, rather than production.

In a consumer society, every producer is in competition with every other
producer. In a replication society, creators benefit from replication.
This creates an incentive system for creators which is the opposite of
the existing one. If I find a way to extract a poison from a river using
simple and readily available materials and transform it into a usable
material for building things, it is in my best interest for that to
replicate. I want other people to copy it because only then will
\emph{all} the water get cleaned up. I want other people to copy it
because the more people copy it the more people will improve upon it,
and I will end up with a vastly superior technology to the one I started
with, making the river I live on cleaner than would have been possible
without the broader community of creators.

Replication societies are nothing new. They are much older than the
production model. Any indigenous society which lives in equilibrium with
its environment is replication based. When people living in a forest use
a tree to make a boat to hunt an animal, that is a replication economy.
They use culture to replicate the boat-making process, and stewardship
of the forest to make sure there are always more trees and animals. The
old boats rot and turn into soil which gets turned back into trees, they
teach their children to pass along the system after they die and turn
back into soil themselves, and the system replicates. What we propose
here is to take this older and proven social model used by indigenous
civilizations and apply it to the materials and principles of modern
machines.

So how do we do this? If all the science needed to build a replication
based civilization exists, why are we not doing it now? To build this
world we must recognize what the hard part is about this. It's not
building the things, we know how to do that. It's not organizing people
to do things, we also know how to do that. It's the replication of the
\emph{desire} to replicate a thing. That is what we call ``magic'' for
the purpose of this work. We use this term because no other term fully
expresses the mystery of the process by which we acquire the desire to
do a thing.

Under the current system, desire to replicate plays a minimal or hidden
role. Most people work for money out of fear, create the systems to do
that out of greed, and consume based on being controlled by a media
which exists for the sole purpose of stimulating consumption. All these
processes are separated from one another. We produce at a ``job'' and
consume separately. Anyone not producing or consuming using money is
treated as a burden on the system.

But if we want technology to replicate freely, we need to harness that
spark which causes a person to suddenly feel a desire to create and
share a new thing. That spark cannot be reduced to rules and numbers and
laws of physics. It is the spark that makes us human, which gives us
free will or the human spirit or soul.

Every single technology we use today relies on this magic. Every
computer, every jet airplane, every factory and medicine--all these
things began with some spark in an actual human which was passed on to
other humans. Technology producers today have mechanistic models for
their technology which ignore life. We use the provocative term
``magic'' to re-center all our thoughts about technology on life itself,
starting with the human spirit and going out to the living world around
us. Life is self-replicating and we identify this word ``magic'' with
all living things. We reject any model for our world which is not
centered on the magic of life itself.

So where does this leave us? We want to transition back to a replication
society while retaining the most useful modern technologies. We are
currently trapped in a system based on scarcity that no one can leave.
So how do we get from one to the other? We must first recognize that the
most powerful engine of change in modern society is social networking.
Working alone, any technology we create is of almost no use. Everything
we create requires that we find ways to collaborate and find people to
share with. The core technology which structures all other technology is
how we communicate with one another in complex networks. If we want to
build a radically different world, we must therefore build a radically
different social network. This work represents the creation of a social
network for the sole purpose of empowering this replication economy.

The transition from a consumer to replication society means replacing
the ``means of production'' with the ``means of replication'' as the
fundamental element of our model of society. Of course we will still
have machines that build more machines and people operating those
machines just as we do today. But we recognize that the most fundamental
thing is not those machines but the social network which tells others
how to build those machines and more importantly \emph{why} they should
build those machines. This transmission of the ``why'' is what makes the
process require our use of the term ``magic''.

In order for this to work we need to have media which supports
self-replicating documents which tell us how to replicate technology,
and this media itself must be self-replicating. This means the media
must carry on it documents which in addition to copying freely from one
device to the next must tell people how to copy the actual physical
devices.

Once this process gains momentum we can use it as the basis of a whole
new economy which allows us to progress into a full replication system.
However, initially we are back to the problem of trying to survive
without money in a system which literally won't let us live without it.
Our way out of this is with a middle path in which we build social media
on hardware which can be bought cheaply and given away to the community
as a free resource for very little money and with no material input from
any central entity like a company. In order for this to scale, each time
someone copies the system it must provide more value added up in
monetary units than it costs to build the copy, including the labor to
put it together.

This is much easier than it sounds. Social media today is a centralized
form of power which generates trillions of dollars in commerce, all
based on software. That software has its replication deliberately
crippled by intellectual property so that a very small group of digital
landlords can take money from everyone else in the system. They get away
with this because of the very real value created by linking us all up
with one another in complex ways. From ride sharing to finding friends
to selling and buying things, all commerce can be dramatically enhanced
by social networking. If it costs under 1000 dollars to build out a
local social network of free book-like documents for a community, all we
need is for that to provide 1000 dollars of value and it will replicate.
In even a small and poor community this is an infinitesimal fraction of
the available commerce which can be amplified by social media.

We do not aim to build a ``new social media platform''. We aim to build
hundreds of millions of truly independent social media platforms, all of
which simply replicate documents from one to another, and all of which
exist for the primary purpose of building out the replication economy
which will transition us off of consumption.

To do this in the long run we will rebuild the hardware from the ground
up along the principles of Geometron laid out here. But we can't get
there until we have a network which is financially self-sustaining in
the existing system. At its core this means finding a way to harness the
``magic'', the core spark which makes a person want to engage with a
thing.

Geometron is a way of thinking about technology in which we think of all
technology as a geometric construction. Shaping metal into machines is a
geometric construction. Displaying symbols on a screen or on paper is a
geometric construction. Printing electronic circuits on a board or chip
is a geometric construction. All the symbols we use to communicate with
one another are geometric constructions. In Geometron we rethink how we
program and control machines based on this idea that geometric
constructions are more fundamental than those using numbers. Numbers are
useful tools, but we choose as a matter of values to always place them
in a subordinate role to geometry.

This is the origin of the term ``Geometron'': ``geometry'' combined with
the ``-tron'' ending which is associated with machines. The work here
demonstrates using this method of geometric programming to create a wide
variety of useful things. We replace ``computer programs'' made up of
numbers, algebra and broken English with geometric constructions
represented with geometric symbols.

This book is therefore combining ``Geometron'' with ``magic'' by
building social media based on these ideas about self-replicating media
and geometric thinking, together in a combined whole. The previous book
of Geometron was more technical and also less action oriented. This work
intends to both tell you why we need to build this system and very
specifically how you can immediately engage by copying parts of it and
recruiting other people to copy more of it. We are asking readers to
\emph{participate} actively through direct action. We are asking you to
tell people about this, to share these ideas and build on them. We are
asking you to help us build a world of abundance from the bottom up
through direction action.

Finally, we must address the problematic connotations of the word
``magic'' and why we choose to use it anyway. Many people of many
religious beliefs either view ``magic'' as a word referring to the
``superstition'' of other people's beliefs or some literal power in the
physical world outside of the realm of science. In this work we are
using the word to refer to an observable phenomenon in the world which
everyone agrees happens, and which applies to everyone's religious,
spiritual, or philosophical beliefs. Every belief, be it one called
``religion'', ``philosophy'', ``ideology'', or ``science'', is based on
this replication from person to person. Beliefs are held by mortal
people. We all die. Our minds decay. We forget. What brings beliefs of
\emph{all} kinds to life is their replication. And this is never just
the mechanical process of printing or broadcasting media or preaching in
person. It is the process that happens deep inside us in which each of
us shifts our internal reality, our internal \emph{desires} in a way
which accommodates the new belief system. This can and does happen with
everything we believe. While one person might not believe in the god of
another, none of us should deny the basic observable phenomenon by which
the other person's belief replicated to get to the point that they
believed it. We can call that replication a ``magic'' without
attributing to it any supernatural connotation.

We hope that people of all faiths can use this framework in a productive
way to build new ways of life. Geometron and Trash Magic are not
religious systems. They are systems of organizing information and
materials which can be fit into any existing religion. In order for this
movement to work we must find ways to be compatible with all existing
faiths, and not to attempt to convert to some new faith. Just as we find
all faiths using oil and mine based supply chains today, and
communicating their faiths via the Internet, we will build a world where
all those people are able to carry on their lives in a post-extraction
framework without creating contradictions with the other parts of their
cosmology separate from the mines and pipelines we seek to replace. This
does not make Geometron and Trash Magic ``non religious'' or
``religious'', but occupying a different space in the human mind and
experience and being boosted by the replication of what is already being
replicated. We must therefore build many ``magics'' which are compatible
with all existing human belief systems, and which can replicate along
with them in their institutions.
